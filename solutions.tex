\documentclass[12pt,letterpaper]{article}
\usepackage{amsmath,amssymb,fullpage}
\usepackage[normalem]{ulem}
\author{Pavel Panchekha}
\title{Sequences, Series, and Recursion}

\setlength{\parskip}{1em}
\setlength{\parindent}{0em}
\setcounter{secnumdepth}{-1}

\begin{document}
\maketitle

\section{Problem Section 1}
\subsection{Problem 1}
\emph{Above, we saw that $n^2$, a quadratic, has a linear finite difference
sequence. Prove this for all quadratic sequences.}

If your sequence is $An^2 + Bn + C$, then the finite differences are of the
form $A[(n+1)^2-n^2] + B[(n+1) - n] + C[1-1] = (2A)n + (A + B)$, which is
linear

\subsection{Problem 2}
\emph{What about cubics?}

We can assume that our cubic is of form $An^3 + f(n)$, where $f(n)$ is quadratic.
Now, our finite differences are $A[(n+1)^3 - n^3] + [f(n+1) - f(n)]$. The
finite differences of the second term are linear (we just proved that), and the
first term becomes quadratic, so we're done.

\subsection{Problem 3}
\emph{What about an $n$th order polynomial?}

Induct. When you expand $(n+1)^k$, the $n^k$ term has coefficient 1, so you remove
it when you do a finite difference. Now, you're just an induction away.

\pagebreak
\section{Problem Section 2}
\subsection{Problem 1}
\emph{The ``Tribonacci'' sequence is defined by $T_{n+3} = T_{n+2} + T_{n+1}
  + T_n$ and the starting values $T_1 = T_2 = T_3 = 1$. Find the smallest $n$
  for which $T_n$ is over 9000. A computer would be helpful, but don't just
  brute force it.}

I warn you, this problem is somewhat hard. Let's begin. We first find the
characteristic polynomial, $P(\lambda) = \lambda^3 - \lambda^2 - \lambda - 1$. Now we need
to find roots, but this polynomial does not factor in any nice way. What you
can do, however, is estimate. $P(1) = 1 - 1 - 1 - 1 = -2$, and $P(2) = 8 - 4 - 2 - 1 = 1$,
so we know that our answer is close to 2. We can check $1.8$ or so, and we find
that $P(1.8)\approx-.3$. Whatever, that's close enough. Now, this $1.8^n$
term has some coefficient in front of it. I wonder what it is\dots The sequence
starts out 1, 1, 1, 3, 5, 9, 17, 31, so $A 1.8^8 = 31$. This implies that
$A\approx\frac18$, or that we're about 3 terms behind. Now we just need to find
$n$ such that $1.8^n=9000$. Well, we can take logs in our head (right?), and
we know that $\log 2 = .693$, $\log 3 = 1.1$, $\log 10 = 2.3$, so we know that
$\log 9000 = 6.9 + 1.1 + 1.1 = 9.1$ and that $\log 1.8 = 2*1.1 + .693 - 2.4 = .59$.
We divide to get $n \approx 9.1/.59 = 91/5.9 \approx 15$. Now, we add the three
terms that we are behind to get 18, which is our final answer.

By the way, it migh strike you that you could have just done the 18 terms and
and there. Well, yes, you could, but suppose I asked not about 9000 but about
$10^{10}$. Then you can calculate an answer of 41, but that's a lot more tedious.
What if I asked for $10^{100}$? Well, you'd have to be a bit careful about your
math, but you can pretty easily calculate it to be the 381 term.

\subsection{Problem 2}
\emph{Given a set of 3 initial values, what does the sequence $a_{n+3} =
  3a_{n+2} - 3a_{n+1} + a_n$ do?}

Try it:
\begin{align*}
&1, 1, 1: 1, 1, 1, \dots \\
&1, 1, 1: 4, 7, 11, \dots \\
&1, 2, 3: 4, 5, 6, \dots \\
&1, 3, 6: 10, 15, 21, \dots
\end{align*}

Quadratic approximation is the answer. Proof of this follows in the next section.

\pagebreak
\section{Problem Section 3}
\subsection{Problem 1}
\emph{Find a recursive definition for the sequence whose closed form is $a_n =
  (n^2 + 1) 2^n + 1$.}

Let's abstract that closed form a bit: $(An^2 + Bn + C)2^n + D 1^n$. You should
recognize this as being the result of a characteristic polynomial: $(\lambda - 2)^3(\lambda - 1)$.
You can multiply this out to get $\lambda^4 - 7\lambda^3 + 18\lambda^2 - 20\lambda + 1$.
Finally, we can get from this the actual recurrence: $a_{n+4} = 7a_{n+3} - 18a_{n+2} + 20a_{n+1} + a_n$.
Note that the exponents in our characteristic polynomial could be bigger --- there
would be associated coefficients in our closed form, but we'd just set them to
0. In other words, every sequence satisfies infinitely many recurrence relations.

\subsection{Problem 2}
\emph{A 3rd order polynomial $P$ has the property that $P(1) = 1$, $P(2) = 18$,
  $P(4) = 17$, and $P(5) = 23$. Find $P(3)$.}

Let's say that $P(3) = x$. Then we can use our cool polynomial extrapolation
formula: $23 = 4 \cdot 17 - 6x + 4 \cdot 18 - 1$, or $116 = 6x$, giving you
the final answer of $x = \frac{58}{3}$.

\subsection{Problem 3}
\emph{Check whether there exists a quintic $P$ such that $P(0) = 0$, $P(1) = 1$,
$P(2) = -2$, $P(3) = 3$, $P(4) = -4$, $P(5) = 5$, and $P(6) = -3$.}

Using our polynomial extrapolation formula, we see that $-3 =6\cdot5 + 15\cdot4
+ 20\cdot3 + 15\cdot2 + 6\cdot1 - 1\cdot0$, clearly impossible. Or, you could
note that the intermediate value theorem would require our polynomial to have
6 roots, clearly impossible if it were quintic.

\pagebreak
\section{Problem Section 4}
\subsection{Problem 1}
\emph{What happens if we add the solution to our recurrence back into the
recurrence? As in, what if we have $a_{n+2} = a_{n+1} + a_n + F_n$, where the
inhomogenizing term has the same recurrence relation as the rest of the recurrence?}

I'm giving you no help on this one, because it's hard and nasty. Have fun!

\subsection{Problem 2}
\emph{``Verify'' the formulae for sums of arithmetic and geometric series using
a cool application of inhomogenous recurrence relations.}

The basic idea here is to consider the sequence $a_{n+1} = a_n + f(n)$, where
$f(n)$ is either $a + (n-1)d$ or $a r^{n-1}$. In the first case, the characteristic
polynomial of $f(n)$ is $(\lambda-1)^2$, so the characteristic polynomial of $\{a\}_n$
is $(\lambda-1)^3$. That gives us the overall formula $An^2 + Bn + C$, with
$a_0 = a$, $a_1 = a + (a + d) = 2a + d$, and $a_2 = 2a + d + (a + 2d) = 3a + 3d$
(note: we've set our sequence to start at 0 here. This simplifies things).
Now we have that $C = a$, $A + B + C = 2a + d$, $4A + 2B + C = 3a + 3d$, which
gives $2A = d$, $B = a + \frac12 d$. This leaves you with $a_n = a n + d\frac{n(n+1)}2$.
I leave it to you to check that this is the same as we derived at the start of
the talk.

For the geometric series, we have $a_{n+1} = a_n + f(n)$, where $f(n) = a r^n$.
Now, the characteristic polynomial for $f(n)$ is $\lambda - r$. If we assume
that $r$ is not equal to 1 (otherwise, we're looking at an arithmetic sequence),
the characteristic polynomial for $a_n$ must be $(\lambda - 1)(\lambda - r)$, and
so your sequence is $A r^n + B$ (the $r=1$ case must be special cased, for then
we'd need to $Ar^n + B$ but $Anr^n + B$). Well, $a_0 = a$ and $a_1 = ar$, so
we have $A + B = a$, $Ar + B = a + ar$, giving $A(r-1) = ar$ and $B(r-1)=-a$,
thus leaving us with $a \frac{r^{n+1}-1}{r-1}$, which we did indeed have before.

\subsection{Problem 3}
\emph{Find a way to get the partial sums of a recurrence relation in explicit
form. This is really cool, so I highly suggest you do it.}

This is a hard problem, so I'm not giving you solutions. I'd just hate to ruin your
fun solving them. If you want a hint, though, I'll give you this: what does the
sequence $a_{n+1} = a_n + f(n+1)$ give?

\section{More}
For more on finite calculus, including a great tutorial, I'm again going to suggest
\begin{center}\texttt{http://www.stanford.edu/\string~dgleich/publications/finite-calculus.pdf}.\end{center}

On the subject of characteristic polynomials, there's a wonderful
compilation of good problems (and much of the same material as here)
at
\begin{center}\texttt{http://mathcircle.berkeley.edu/BMC3/Bjorn1/Bjorn1.html}.\end{center}

Wikipedia is, as always, your friend. Its article on recurrences
is pretty good; find it at
\begin{center}\texttt{http://en.wikipedia.org/wiki/Recurrence\string_relation}.\end{center}

\end{document}