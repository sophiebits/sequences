\documentclass[12pt,letterpaper]{article}
\usepackage{amsmath,amssymb,fullpage}
\usepackage[normalem]{ulem}
\author{Pavel Panchekha \thanks{Thanks to Ben Alpert and Ben Kraft for reading and revising drafts.}}
\title{Sequences, Series and Recursion}

\setlength{\parskip}{1em}
\setlength{\parindent}{0em}
\setcounter{secnumdepth}{-1}

\begin{document}
\maketitle

\section{Formalisms}
Sequences are a common topic of math competition questions and, in
general, are something you should know about. Formally, a sequence $a$ or
$\{a\}_n$ is a ordered sequence of numbers $a_1, a_2, a_3, ...$, in
principle infinite. Depending on who you ask, a sequence starts at
$a_0$ or $a_1$ (we'll be using $a_1$ here, as that is the common
mathematical convention). Often, elements of a sequence are integral,
but that isn't necessary. Usually, a sequence is generated by a rule,
but neither is that necessary. Most of you already know all of this.

We can also define the sum of the first $n$ elements of a sequence as
$a_1 + a_2 + a_3 + \cdots + a_n$, denoted by $\sum_{k=1}^n a_k$.
Lastly, we can define the infinite sum $\sum_{k=1}^\infty a_k$, or
just $\sum_k a_k$, as the limit of the partial sums.

\section{Basic Sequence Types}
The most basic sequence types are arithmetic and geometric sequences.
An arithmetic sequence is given recursively by $a_n = a_{n-1} + d$,
and a geometric sequence by $a_n = a_{n-1} r$, with $a_1$, $r$, and
$d$ given. It's pretty easy to see the closed-form (non-recursive) way
to represent these: in an arithmetic sequence, we add $d$
each time we go to the next element, we see that $a_n = a_1 + (n - 1)
d$, and for a geometric sequence, we use the same reasoning to arrive
at $a_n = a_1 r^{n - 1}$.

Now, what about sums of these sequences? Well, for an arithmetic
sequence, we can take the elements two at a time, with the first and
last being a pair, then the second and second-to-last, and so
on. Notice that all of these pairs have the same sum. Now, if we have
an even number of elements, all the numbers will be paired up, and each pair will have
sum $a_1 + (a_1 + (n - 1) d) = 2 a_1 + (n - 1) d$. There will be
$n/2$ such pairs, so the final sum is

$$a_1 n + d \frac{n (n-1)}{2}.$$

On the other hand, if we have an odd number of terms, we have
$(n-1)/2$ pairs of size $2 a_1 + (n - 1) d$ and a singleton of
half that size, so the final result is

$$\frac{n-1}{2}(2a_1 + (n-1)d) + \frac{1}{2}(2a_1 + (n-1)d) \\
= a_1 n + d \frac{n (n-1)}{2},$$

which is the same as what we got for even $n$. Cool!

Now, for geometric sequences, we use the fact that $\frac{r^n - 1}{r
  - 1}$ is equal to $1 + r + r^2 + r^3 + ... + r^{n-1}$. From this, we
see that the sum of a geometric sequence is $a_1 \frac{r^n - 1}{r -
  1}$. An important note is that in the case of an infinite sum, this
is equal to $a_1/(1 - r)$.

But anyway, geometric and arithmetic series are boring. Let's move on to
something more fun!

\section{Recursively Defined Sequences}
In the very general case, recursively defined sequences are those of
the form $$a_n = f(a_1, a_2, a_3, \ldots, a_{n-1}) + g(n).$$ If $g(n) = 0$,
we call such a recursive definition homogeneous; if $f$ doesn't depend
on anything except the last $k$ terms, we say the definition is
$k$th-order. In general, however, we only care about two very
specific classes of recursive definitions: $k$th-order linear recurrences and
everything else. A $k$th-order linear recursive equation is one of
the form:

$$a_n = b_1 a_{n-1} + b_2 a_{n-2} + ... + b_k a_{n - k} + g(n).$$

Generally, we try as best we can to avoid non-homogeneous
relations. Now, let's look at a problem that has to do with linear
recurrence relations of this sort. This problem is rather old (first stated
in Europe in 1202), but it is nonetheless a
good example of the techniques that we'll be using.

You have the following model of rabbit growth: At first, you start
with one immature rabbit, and each month, all immature rabbits mature
and all mature rabbits give birth to an immature rabbit (somehow, they
reproduce asexually, but that may have something to do with the fact
that they're immortal).

Now, we can get the first few terms (always a good idea): $1, 1, 2, 3,
5$.  In fact, we notice that each month, the number of mature rabbits
is the number of rabbits the month before. Thus, we quickly derive the
recurrence relation $R_{n+2} = R_{n+1} + R_n$. Now, this is cool, and
we can quickly get lots of terms, but how about a closed form?

To find a closed form, we pretend that the closed form expression is of the
form $R_n = \lambda^n$. Now, substituting in, we see that

\begin{align*}
\lambda^n & = \lambda^{n-1} + \lambda^{n-2} \\
\lambda^2 & - \lambda - 1 = 0 \\
\lambda & = \dfrac{1 \pm \sqrt{1 - 4 (-1)}}{2} \\
        & = \dfrac{1 \pm \sqrt{5}}{2}.
\end{align*}

Now, we'll call the positive case $\phi$, and we note that the minus
case is really just $\frac1{-\phi}$. Now we plug in these values, and get
that $R_n = A \phi^n + B (-\phi)^{-n}$. Since we know the starting
values ($R_1 = 1$, $R_2 = 1$), we can plug in and solve for $A$ and
$B$:

\begin{align*}
  A \phi + B (1 - \phi) & = 1 \\
  A (1 + \phi) + B (2 - \phi) & = 1 \\
  B + (A - B) \phi & = 1 \\
  A + 2B + (A - B)\phi) & = 1 \\
  A + B & = 0 \\
  B - 2B \phi & = 1 \\
  B = \frac{1}{1-2\phi} & = \frac1{\sqrt{5}}
\end{align*}

There's a helpful trick here, however: for a sequence of the second
order, with starting values $a_0 = 0$ and $a_1 = 1$, both coefficients
are the same, and both are equal to $\frac1{\sqrt{D}}$, where $D$ is the
determinant of the characteristic polynomial (the polynomial
in $\lambda$ that we get after dividing out $\lambda^{n-2}$). Now, our
sequence does not have an $R_0$ term, but we can extend it
backwards. Since $R_2 = 1$ and $R_1 = 1$, we know $1 = 1 + R_0$ and $R_0 = 0$.
Now, we easily see that the equation for the sequence is

$$R_n = \frac{\phi^n - (-\phi)^{-n}}{\sqrt{5}}.$$

By the way, you should all recognize this as the Fibonacci sequence. We
can make the formula we derived even nicer by noticing that
$(-\phi)^{-1}$ term is less than $1/2$ for $n \ge 2$, so we can just ignore
it and get the simpler formula $$R_n \approx
\frac1{\sqrt{5}} \phi^n.$$

\section{Problem Section 1}
\begin{enumerate}
  \item The ``Tribonacci'' sequence is defined by $T_{n+3} = T_{n+2} + T_{n+1}
  + T_n$ and the starting values $T_1 = T_2 = T_3 = 1$. Find the smallest $n$ for which $T_n$ is over
  9000.
  \item Given a set of 3 initial values, what does the sequence $a_{n+3} =
  3a_{n+2} - 3a_{n+1} + a_n$ do?
\end{enumerate}

\section{Multiple Roots and Polynomial Approximations}
In general, given a linear, homogeneous sequence, we can form the
characteristic polynomial by rewriting the sequence but replacing
$a_{n+k}$ with $\lambda^k$. Finding the roots of the polynomial
$r_0, r_1, ..., r_k$ lets us write the closed form expression in the
form $a_n = C_0 r_0^n + C_1 r_1^n + ... + C_k r_k^n$. One important
thing to note is that if one of the roots is a double root or
$k$th-order root, the corresponding coefficient will instead be a $(k-1)$-order
polynomial.

Here's an example where the roots can have multiplicities. Given a
sequence $a$ where $a_{n+2} = 2a_{n+1} - a_n$, describe what
this sequence does for starting values $a_1$ and $a_2$.

Well, we can try an example or two (always a good idea!). If the starting
values are $a_1 = 2$ and $a_2 = 7$, we have the sequence $2, 7, 12, 17, \ldots\,$.
If the starting values are $a_1 = 3$ and $a_2 = 5$, we have $3, 5, 7, 9, 11, \ldots\,$.
In general, the sequence looks like it's making an arithmetic sequence from
the starting values. Let's see why.

We'll use characteristic polynomials. Forming the characteristic polynomial,
we get $\lambda^2 = 2\lambda - 1$. We instantly see that this factors, giving
$(\lambda - 1)^2 = 0$, so 1 is a double root. Now, since it's a double root,
the closed form we get will have a coefficient in front of $r_0^n$ that is
linear (the double root means the coefficient will have two terms, meaning it's linear).
So, our closed form looks like $a_n = (An + B)1^n$, which obviously simplifies to $a_n = An+B$.
Thus, we've proven that our sequence from before produces a linear extrapolation
of the starting terms.

In fact, you can generalize this to get a quick way to do $n$th-order extrapolations
of a few data points. If we want a cubic approximation to $1, 2, 5, 3$, we simply
use the recursive sequence $a_{n+4} = 4a_{n+3} - 6a_{n+2} + 4a_{n+1} - a_n$. So we
can quickly calculate that the next term would be $4 \cdot 3 - 6 \cdot 5 + 4 \cdot
2 - 1$ $= 12 - 30 + 8 - 1$ $=-11$.

This also proves the nontrivial property that if the value of an
$n$th-order polynomial is integral at $n+1$ consecutive points, it is integral
for any integer argument.

\section{Problem Section 2}
\begin{enumerate}
  \item Find a recursive definition for the sequence whose closed form is $a_n =
  (n^2 + 1) 2^n + 1$.
  \item A 3rd order polynomial $P$ has the property that $P(1) = 1$, $P(2) = 18$,
  $P(4) = 17$, and $P(5) = 23$. Find $P(3)$.
  \item Check whether there exists a quintic $P$ such that $P(0) = 0$, $P(1) = 1$, $P(2) = -2$,
  $P(3) = 3$, $P(4) = -4$, $P(5) = 5$, and $P(6) = -3$.
\end{enumerate}

Here's another cool use of characteristic polynomials. Let's say we have a
sequence which is periodic with period $p$. Then your recurrence is
$a_{n+p} = a_n$, and your characteristic polynomial $\lambda^p-1=0$.
The roots of this are the $p$-th roots of unity (which is, honestly,
pretty cool by itself). It also means that a complete
characterization of periodic functions (periodic on the integers, that
is) is just that: a linear combination of terms involving $e^{2\pi il/p}$. This has
cool connections to number theory and group theory, but let's get back
to the main topic of this talk.

So, let's do another problem. We have the sequence $1, 2, 4, 8, 16, \ldots, 2^n$. However,
we don't recognize that the sequence is just the powers of two (silly us!), so when we are
asked for the next term, we just do a polynomial approximation. How
far off are we?

Let's try a few simple examples. If we call the $P(n)$ the value we're trying to find, we
see that:
\begin{align*}
P(1) &= 1 = 1 \\
P(2) &= 2\cdot2-1 = 3 \\
P(3) &= 3\cdot4-3\cdot2+1 = 7 \\
P(4) &= 4\cdot8-6\cdot4+4\cdot2-1 = 15
\end{align*}
Note that in each case, $P(n)$ is exactly one less than $2^n$. Let's try to prove it.

Now, since we have a shiny new hammer (ooh, characteristic polynomials\ldots\ shiny!), let's try
to hit this problem on the head with it. We know that we're going to need the value of $$
\sum_{k=0}^{n} (-1)^n \binom{n+1}{n-k} 2^{n-k}.$$ How do we do it? Let's simplify a bit, by
subtracting this from $2^n$: $$2^{n+1} - \sum_{k=0}^{n+1} (-1)^k \binom{n+1}{k} 2^{n+1-k}.$$

Notice anything? It looks like the binomial theorem: the above is equivalent to $(-1 + 2) ^ k = 1$,
so in every case, we are off by exactly 1, in that we always end up with something one less
than the correct value. This proof, by the way, though slick, is
rather opaque -- the reason the value is one away is rather non-obvious.

\section{Finite Differences}
Another good tool to have to attack sequences, one that is much
simpler than characteristic polynomials, is finite differences.
Basically, the finite differences of a sequence $a_n$ are the
sequence $b_n = a_{n+1} - a_n$. These can give insight into how a
sequence works.

For example, if you have the sequence $1, 4, 9, 16, 25, \ldots\,$, the
finite differences make the sequence of odd numbers $3, 5, 7, 9, \ldots$ (prove it!).
The finite differences of this are just $2, 2, 2, \ldots\,$. Now, there are
a few obvious properties. Firstly, any linear sequence has
constant finite differences (prove it!). Also, if you have the
geometric sequence $a_n = a_0 r^n$, the finite difference is
$b_n = a_0 (r-1) r^n$ (prove it!).

\section{Problem Section 3}
\begin{enumerate}
\item Above, we saw that $n^2$, a quadratic, has a linear finite difference
sequence. Prove this for all quadratic sequences.
\item What about cubics?
\item What about an $n$th order polynomial?
\end{enumerate}

\section{Rewind}
So, let's go back to the problem we had before, on the estimation of
$2^n$. We'll be using the result that you end up proving in the
problem section just before, so you should probably do it. It's not
too hard. (Really. It's not. Go do it.)

Now then, since we know that an $n$th order polynomial will become
an $(n-1)$-th order polynomial when you take its finite differences, we can
``work backwards'' to get a polynomial approximation to a sequence: if
you have $n$ terms, $n-1$ finite differences will reduce you to a
single number, which obviously has a constant polynomial
approximation. You can then extend that sequence, and work backwards.

This actually leads to a trivial proof of what we had above. Consider
the finite differences of $1, 2, 4, \ldots, 2^n, 2^{n+1}-1$. What is
it? $1, 2, 4, \ldots, 2^{n-1}, 2^n-1$ (check it!)! So, all we need is
some trivial induction to prove our claim above.

\section{Hmm\dots\ This Needs More Calculus}
Doesn't everything?

Now, if we have a function $f(n)$, we can define the \emph{finite
derivative} $\frac{\Delta f}{\Delta n}$ to be $f(n+1)-f(n)$, that is,
the finite differences of the sequence $a_n = f(n)$. You'll note that
this definition is somewhat similar to the calculus definition:
$$\frac{df}{dx} = \lim_{h\to0}\frac{f(x+h) - f(x)}{h},$$ but when
we say $\lim_{a\to b}$, we mean ``get $a$ as close as you can to $b$
without getting there''. And in the integers, the closest you can get
is one away, so the normal derivative becomes the finite
derivative.

Why is this useful? Well, it's really cool, and there's a lot more to
say on this subject than I will\footnote{See:
\texttt{http://www.stanford.edu/\string~dgleich/publications/finite-calculus.pdf}},
but there are a few applications we want to use on our problem above,
to actually explain why the answer is $2^{n+1}-1$.

What is $\frac{\Delta}{\Delta n} 2^n$? It's $2^n$ -- we proved that above
(or at least, you should have). That's actually pretty remarkable,
especially if you know regular calculus, where the equivalent function
is $e^x$ (which is part of why $e$ is so important).

What about $n^2$? We have $(n+1)^2 - n^2 = 2n + 1$. Hmm. We'd like it to be
$2n$, so that it's similar to regular calculus. Note that instead,
$n^2 - n$'s finite derivative is $2n$. What
about $n^3$? What's its replacement? Do we have to get these
experimentally? Is there a general rule? There is. Consider
$n^{\uline k}$ = $n (n-1) (n-2) (n-3) \cdots (n-k+1)$. Prove for
yourself that $\frac{\Delta}{\Delta n} n^{\uline k} = k n^{k-1}$.

Now, I really want to get back to resolving our problem, but I'll
return to other cool applications of finite calculus. For now,
though, I'll make an unqualified statement: Taylor's theorem works in
finite calculus (the proof is straightforward if you know the proof
in regular calculus; it is, however, tedious). So, what does that
mean? It means there's an operator $T_k$, where $T_k f(n)$ is defined as
$$T_k f(n) = \sum_{l=0}^{\infty} \frac{(\frac{\Delta}{\Delta n} f)(k)}{l!}
(n-k)^{\uline l},$$ and that (provided certain conditions are met) $T_k f(n)$
is equal to $f(n)$ (at least, in a region near $k$). Now, simplifying
the above a bit and setting $k$ to be 0, we get $$\sum_{l=0}^\infty
\binom{n}{l} \left(\frac{\Delta}{\Delta n} f\right)(0).$$

Isn't it cool that the binomial coefficients just popped out of
nowhere? It's also crucial to our problem. As in regular Taylor's
theorem, there's a sense in which $T f(n)$, if you cut it down to the
first $k+1$ terms, is the best $k$th order polynomial approximation
to $f(n)$. So, to solve the above problem, we just have to consider
the Taylor expansion of $2^{n+1}$, with the $(n+2)$-th term chopped off.

What's the Taylor expansion of $2^{n+1}$? Since $\frac{\Delta}{\Delta
n} 2^n = 2^n$, and $2^0 = 1$, it's $\binom{n}{0} + \binom{n}{1} +
\binom{n}{2} + \ldots$, which is indeed equal to $2^n$. Now, this
series cuts itself off at some
point (at $n$ terms), which is a very nice feature -- you don't have
to chop off infinitely many terms. Just one will do.

So, we have our polynomial approximation to $2^{n+1}$:
$$2^{n+1} \approx \sum_{k=0}^{n} \binom{n+1}{k}.$$ Note that we're not adding in the
last term here. But what is that last term? $\binom{n+1}{n+1} = 1$,
which is exactly why we're 1 off in our approximation. Now we, in a
sense, know the \emph{reason} for the theorem we've been proving. This
actually also makes it very easy to extend our solution. What if we
extend the approximation \emph{two} terms ahead? Well, we're chopping
off $\binom{n+1}{n} + \binom{n+1}{n+1}$, or $n+2$, so we'll be that
far off. Extend $1, 2, 4, 8, 16, 32$ two terms out, and you're
going to get $63$, then $121$. Isn't that cool?

\section{More}
For more on finite calculus, including a great tutorial, I'm again going to suggest
\begin{center}\texttt{http://www.stanford.edu/\string~dgleich/publications/finite-calculus.pdf}.\end{center}

On the subject of characteristic polynomials, there's a wonderful
compilation of good problems at and much of the same material as here
at
\begin{center}\texttt{http://mathcircle.berkeley.edu/BMC3/Bjorn1/Bjorn1.html}.\end{center}

Wikipedia is, as always, your friend. Its article on recurrences
is pretty good; find it at
\begin{center}\texttt{http://en.wikipedia.org/wiki/Recurrence\string_relation}.\end{center}

\end{document}
